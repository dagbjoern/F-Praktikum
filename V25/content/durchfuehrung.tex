\section{Aufbau und Durchführung}
\label{sec:Durchführung}
Das zu untersuchende Material, $^39K$, wird in einem Ofen bei $190\si{\degree}$
verdampft und durch ein Blendensystem gebündelt und fokussiert.
Die Temperatur wird mittels CuNi-Thermoelement überprüft.
Nach dem Ofen folgt ein Elektromagnet mit zylinderförmigen Polschuhprofil.
Ein Langmuir-Taylor Detektor detektiert die austretenden Teilchen.
Der Detektor ist auf einer Schiene gelagert und kann so zum Vermessen der Intensitätsverteilung einen Bereich abfahren.
Die Apparatur befindet sich im Vakuum um Stöße mit Fremdatomen zu vermeiden.

Vermessen wird die Nullposition und die Intensitätsverteilung für sieben verschiedene Magnetfeldgradienten.
