\section{Aufbau und Durchführung}
\label{sec:Durchführung}
Der Versuchsaufbau ist bereits in der Theorie näher beschrieben.
Zusätzlich sei erwähnt, dass die Apparatur im Vakuum gelagert ist, um Stöße mit Fremdatomen zu vermeiden.
Ebenfalls ist der Detektor auf einer Schiene gelagert und kann so zum Vermessen der Intensitätsverteilung
einen Bereich abfahren.\\
Das zu untersuchende Material, $^{39}K$, wird zu Beginn in einem Ofen bei $190\si{\degree}$
verdampft. Die Temperatur wird dabei mittels eines CuNi-Thermoelement überprüft.
Vermessen wird die Nullposition und die Intensitätsverteilung für sieben verschiedene Magnetfeldgradienten.
