\section{Aufbau und Durchführung}
\label{sec:Durchführung}
Das zu untersuchende Material, $^39K$, wird in einem Ofen bei $190\si{\degree}$
verdmpft und durch ein Blndensystem gebündelt und fokussiert.
Die Temperatur wird mittels CuNi-Thermoelement gemessen.
Auf den Ofen folgt ein Elektromagnet mit zylinderförmigen Polschuhprofil.
Ein Langmuir-Taylor Detektor detektiert die austretenden Teilhen.
Die Apparatur befindet sich im Vakuum um Stöße mit Fremdatomen zu vermeiden.
