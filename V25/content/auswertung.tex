\newpage
\section{Auswertung}
\label{sec:Auswertung}
Es wird die Spannung am LT-Detekor für unterschiedliche Position des LT-Detektors, welche in Spindelumdrehungen $Skt$ angegeben wird, gemessen.
Dabei ist die Spannung an einer Position proportional zu der Intensität des Strahles an ebendieser.
Zusätzlich wird der Strom in dem Ablenkmagneten varriert, sodass sich der
magnetische Feldgradienten $\frac{\partial B}{\partial z}$
verändert. Die Messwerte sind in den Kapitel \ref{sec:mess} aufgelistet.
Um aus den Messwerten das Bohr'sche Magneton zu berechnen,
werden für verschiedene Feldgradienten $\frac{\partial B}{\partial z}$ die
Messwerte der Spannung am LT-Detekor in Abhängigkeit der Position aufgetragen.
Tabelle \ref{tab:strom} enthält die verwendeten Ströme für den
Ablenkmagneten und die entsprechenden magnetischen Feldstärken $B$. Aus der
Gleichung \eqref{eqn:grad}
% \begin{align}
%   \frac{\partial B}{\partial z}=0,968\frac{B}{a} \label{eqn:grad}
% \end{align}
lässt sich der entsprechende Feldgradient $\frac{\partial B}{\partial z}$ berechnen.
\begin{table}
\centering
\caption{Einstellung der Ströme und B-Felder und B-Feldgradienten aus Gleichung \eqref{eqn:grad} für die verschiedenen Messungen.}
\label{tab:strom}
\begin{tabular}{c c c}
\toprule
 $I/\si{\milli\ampere}$ & $B/\si{\tesla}$ & $\frac{\partial B}{\partial z}/\si{\tesla\per\meter}$\\
\midrule
0     &  0     &  0        \\
0,4   & 0,3    & 116,16    \\
0,5   & 0,38   & 147,13    \\
0,6   & 0,45   & 174,24    \\
0,735 & 0,55   & 212,96    \\
0,8   & 0,595  & 230,38    \\
0,9   & 0,765  & 296,208   \\
\bottomrule
\end{tabular}
\end{table}

In den Abbildungen \ref{fig:mess0} und \ref{fig:mess_all}
sind für die verschiedenen B-Feldgradienten
die Spannungen in $\si{\milli\volt}$ in Abhängigkeit von der Position des LT-Detektors dargestellt.

\begin{figure}
  \centering
  \includegraphics{build/plot_B=0.pdf}
  \caption{Messung für $\frac{\partial B}{\partial z}=0$ }
  \label{fig:mess0}
\end{figure}

\begin{figure}
   \centering
   \begin{subfigure}{0.48\textwidth}
      \includegraphics[width=1\textwidth]{build/plot1.pdf}
      \caption{Messung für $\frac{\partial B}{\partial z}=116,16\si{\tesla\per\meter}$. }
      \label{fig:mess1}
     \end{subfigure}
     \begin{subfigure}{0.48\textwidth}
       \includegraphics[width=1\textwidth]{build/plot2.pdf}
       \caption{Messung für $\frac{\partial B}{\partial z}=  147,13\si{\tesla\per\meter}$. }
       \label{fig:mess2}
     \end{subfigure}
     \begin{subfigure}{0.48\textwidth}
       \includegraphics[width=1\textwidth]{build/plot3.pdf}
       \caption{Messung für $\frac{\partial B}{\partial z}=174,24\si{\tesla\per\meter}$. }
       \label{fig:mess3}
     \end{subfigure}
     \begin{subfigure}{0.48\textwidth}
       \includegraphics[width=1\textwidth]{build/plot4.pdf}
       \caption{Messung für $\frac{\partial B}{\partial z}=212,96\si{\tesla\per\meter}$. }
       \label{fig:mess4}
     \end{subfigure}
     \begin{subfigure}{0.48\textwidth}
       \includegraphics[width=1\textwidth]{build/plot5.pdf}
       \caption{Messung für $\frac{\partial B}{\partial z}=230,38\si{\tesla\per\meter}$. }
       \label{fig:mess5}
     \end{subfigure}
     \begin{subfigure}[width=1\textwidth]{0.48\textwidth}
       \includegraphics[width=1\textwidth]{build/plot6.pdf}
       \caption{Messung für $\frac{\partial B}{\partial z}=296,208\si{\tesla\per\meter}$. }
       \label{fig:mess6}
     \end{subfigure}
     \caption{Messwerte der Intensitätsverteilung für die unterschiedlichen B-Feldgradienten aus der Tabelle \ref{tab:strom}.}
    \label{fig:mess_all}
\end{figure}

An die Messwerte, um die Intensitätsmaxima aus den Abbildungen
\ref{fig:mess0}-\ref{fig:mess_all}, wird jeweils eine
Gaußfunktion der Form
\begin{align}
  f(x)=a\exp\left({-\frac{(x-b)^2}{2c^2}}\right)+d
\end{align}
gefittet. Der Parameter $b$ ist dabei die
Position des Intensitätsmaximums und dient hier zur Bestimmung
des Abstandes $s$ zu Nullposition die aus
der Abbildung \ref{fig:mess0} zu
\begin{align}
b_0= (6,724\pm0,004 )Skt
\end{align}
bestimmt wird.
Für die Messungen mit $\frac{\partial B}{\partial z}<0$ Dabei wird zwischen
$b_l$ die Position des linken Intensitätsmaximums und
$b_r$ die Position des rechten Intensitätsmaximums
unterschieden. Tabelle \ref{tab:parameter}
enthält jeweils den bestimmten Parameter
$b$ für die unterschiedlichen B-Feldgradienten.

\begin{table}
    \centering
    \caption{Position der Intensitätsmaxima
    aus Abbildung \ref{fig:mess_all} für die
     Unterschiedlichen B-Feldgradienten.}
    \label{tab:parameter}
    \begin{tabular}{c c c}
      \toprule
      $\frac{\partial B}{\partial z}/\si{\tesla\per\meter}$ & $b_l/Skt$ & $b_r/Skt$ \\
      \midrule
      147,13  & 6,189\pm0,004 & 7,337\pm0,002  \\
      116,16  & 6,095\pm0,007 & 7,410\pm0,003  \\
      174,24  & 6,005\pm0,003 & 7,493\pm0,003  \\
      212,96  & 5,879\pm0,004 & 7,587\pm0,009  \\
      230,38  & 5,839\pm0,003 & 7,596\pm0,006  \\
      296,208 & 5,757\pm0,004 & 7,735\pm0,006  \\
      \bottomrule
    \end{tabular}
\end{table}

Tabelle \ref{tab:abstand} enthält
die Abstände $s$ von den Intensitätsmaxima zur Nullposition.
Ebenfalls wird der Abstand zwischen den beiden Intensitätsmaxima bestimmt
und halbiert, um ebenfalls den Abstand $s$ zu bestimmen.
Die bestimmten Abstände $s$ werden gemittelt
(eine Spindelumdrehungen $Skt$
 entspricht dabei \SI{1.8}{\milli\meter}).
\begin{table}
    \centering
    \caption{Abstände $s$ der Intensitätsmaxima zur Nullposition aus Abbildung \ref{fig:mess_all} für die unterschiedlichen B-Feldgradienten.}
    \label{tab:abstand}
    \begin{tabular}{c c c c c c }
      \toprule
      $\frac{\partial B}{\partial z}/\si{\tesla\per\meter}$ & $s_l/Skt$ & $s_r/Skt$ & $\frac{b_r-b_l}{2}/Skt$ & $\bar{s}/Skt$ &$\bar{s}/\si{\milli\meter}$\\
      \midrule
      147,13  & 0,535\pm0,005  & 0,613\pm0,004  & 0,574\pm0,002  & 0,57\pm0,03  & 1,03\pm0,06 \\
      116,16  & 0,629\pm0,008  & 0,686\pm0,004  & 0,657\pm0,004  & 0,66\pm0,02  & 1,18\pm0,04 \\
      174,24  & 0,719\pm0,005  & 0,769\pm0,004  & 0,744\pm0,002  & 0,74\pm0,02  & 1,34\pm0,04 \\
      212,96  & 0,845\pm0,005  & 0,863\pm0,009  & 0,854\pm0,005  & 0,85\pm0,01  & 1,54\pm0,01 \\
      230,38  & 0,885\pm0,005  & 0,872\pm0,007  & 0,878\pm0,003  & 0,88\pm0,01  & 1,58\pm0,01 \\
      296,208 & 0,967\pm0,005  & 1,011\pm0,007  & 0,989\pm0,004  & 0,99\pm0,02  & 1,78\pm0,03 \\
      \bottomrule
    \end{tabular}
\end{table}

Die gemittelten Abstände $\bar{s}$ zu Nullposition
aus der Tabelle \ref{tab:abstand}
werden in Abhängigkeit zu dem Feldgradienten $\frac{\partial B}{\partial z}$ in Abbildung
\ref{fig:lin} aufgetragen, um ein lineare Ausgleichsrechnung mit der Funktion
\begin{align}
s=\mu_{\mathrm{B}}\frac{l\cdot L\left(1-\frac{L}{2l}\right)}{6 k T}\frac{\partial B}{\partial z}
\end{align}
durchgeführt.
\begin{figure}
  \centering
  \includegraphics{build/lin_plot.pdf}
  \caption{Abstände $\bar{s}$ zwischen den Intensiätsmaxima in Abhängigkeit zu dem B-Feldgradienten $\frac{\partial B}{\partial z}$ und Lineare Fit-Funktion. }
  \label{fig:lin}
\end{figure}

Dabei ist $\mu_{\mathrm{B}}$ das Bohr'sche Magneton der zu bestimmende Parameter, $T$ die Ofentemperatur,
$L$ die Länge der Polschuhe und $l$ der Abstand von der Eintrittsblende $B_4$ (Abb.\ref{fig:schema}),
sie betragen
\begin{align}
  T&=\SI{463,15}{\kelvin}\\
  L&=\SI{0,070}{\meter}\\
  l&=\SI{0,455}{\meter}.
\end{align}
Aus der Ausgleichsrechnung folgt für das aus der Messung bestimmte
Bohr'sche Magneton
\begin{align}
\mu_{\mathrm{B}}=\SI{9,1(5) e-24}{\joule\per\tesla }.
\end{align}
