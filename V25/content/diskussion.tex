\section{Diskussion}
\label{sec:Diskussion}
Das Experiment bestätigt die Richtungsquantisierung von Elektronen wobei
es möglich ist durch das Experiment das Bohr'sche Magneton $\mu_\mathrm{B}$
zu bestimmt. Die Tabelle \ref{tab:erg} enthält den bestimmten Wert
und den theoretischen  Wert von $\mu_\mathrm{B}$ sowie die Abweichung
des bestimmten Wertes zu dem Theoriewert.
\begin{table}
  \centering
  \caption{Aus der Messung berechnetes $\mu_\mathrm{B}$ sowie der Theoriewert und die Abweichung $a$ zu dem Theoriewert.}
  \label{tab:erg}
  \begin{tabular}{c c c}
    \toprule
  Messwert $\mu_\mathrm{B} / \si{\joule\per\tesla}\cdot 10^{-24} $ & Theoriewert $\mu_\mathrm{B} / \si{\joule\per\tesla }\cdot 10^{-24}$ & Abweichung $a$/\si{\percent}\\
    \midrule
9,1\pm0,5 & 9,274 & 2\pm5 \\
    \bottomrule
  \end{tabular}
\end{table}
Es zeigt sich das es durch den Stern-Gerlach-Versuch möglich ist, dass
Bohr'sche Magneton $\mu_\mathrm{B}$ bis zu einer Abweichung von \SI{2(5)}{\percent}
zu messen. Bei Bestimmung der Abstände zwischen den Intersitätsmaxima in Abhängigkeit
des B-Feldgradienten $\frac{\partial B}{\partial z}$ (Abb.\ref{fig:lin})
wird jedoch ein abflachender Verlauf der Messwerte festgestellt.
Dieser lässt sich möglicherweise durch die Formel \eqref{eqn:grad}, durch die der B-Feldgradienten $\frac{\partial B}{\partial z}$ berechnete wird,
 erklären, da der Zusammenhang zwischen B-Feldstärke und B-Feldgradient nur Nährungsweise linerar ist.
Dies kann jedoch vernachlässigt werden, da die Nährung ausreicht um das Bohr'sche Magneton $\mu_\mathrm{B}$ zu berechnen.
Um eine genauere Messung zu erreichen, könnte beispielsweise die Anzahl der
verwendeten B-Feldgradienten erhöht werden oder Terme höherer Ordnung in dem B-Feldgradienten berücksichtigt werden.
Alles im allem eignet sich der Versuch, um das Bohr'sche Magneton $\mu_\mathrm{B}$
genau zu bestimmen.
