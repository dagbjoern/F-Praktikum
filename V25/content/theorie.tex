\section{Theorie}
\label{sec:Theorie}
Im ursprünglichen Stern-Gerlach Versuch ist die Aufspaltung eines Silber-Atomstrahls beim
Durchlauf eines inhomogenen Magnetfeldes zu beobachten.
An den Spin vom äußeren Elektron des Silber-Atoms koppelt ein magnetischesn
Moment $\mu$, welches für die Ablenkung verantwortlich ist.
Das Magnetische Moment wird bestimmt durch den Eigendrehimpuls $S$,
den Bahndrehimpuls $S$ und den Kernspin $I$ zu: $\mu=\mu_\mathrm{S}+\mu_\mathrm{L}+\mu_\mathrm{I}$.
Hierbei ist $\mu_\mathrm{I}$ vernachlässigbar klein gegenüber den anderen
magnetischen Momenten.
Bei Atomen mit einem Valenzelektron in der s-Schale ist der Bahndrehimpuls
$L$ gleich null und somit auch $\mu_\mathrm{L}$.
Es bleibt ein magnetisches Moment von:
\begin{align}
  \vec{\mu}_\mathrm{S}=-\frac{e}{2m}g_\mathrm{S}\vec{S}
\end{align}
übrig. Das gyromagnetiche Verhältnis wird mit $g_\mathrm{S}\approx 2$ bezeichnet.
Besteht ein Magnetfeld in z-Richtung, so sind zwei Spineinstellungen möglich und
es gilt für die z-Komponente des magnetischen Momentes:
\begin{align}
  \mu_\mathrm{s_\mathrm{z}}=-g_\mathrm{s}m_\mathrm{s}\mu_\mathrm{B}
\end{align}
mit $\mu_\mathrm{B}=\frac{e}{2m}\hbar$ dem Bohr'schen Magneton.
Bei der Wechselwirkung der Atomen mit einem inhomogenen Magnetfeld
in z-Richtung wirkt auf diese eine Kraft:
\begin{align}
  F_\mathrm{z}=-m\mu_\mathrm{B}\frac{\partial B}{\partial z}
\end{align}
in z-Richtung. Hier beschreibt $m$ die Eigenwerte von $\mu_\mathrm{S}$.

\subsection{}
Der in diesem Versuch genutzte Elektromagnet besitzt ein zylinderförmiges
Polschuhprofil, gezeigt in Abbildung \ref{fig:magnet}.
Dessen Abmessungen
