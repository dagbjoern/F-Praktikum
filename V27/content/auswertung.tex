\section{Auswertung}
\label{sec:Auswertung}
\subsection{Theoretische Werte}
Zu Beginn können mit Hilfe der Formel \eqref{eqn:lande}
die theoretischen Werte der Lande-Faktoren
für die entsprechenden Orbitale ausgerechnet werden.
Die Berechneten Werte sind in der Tabelle \ref{tab:theo1}
zu finden.

\begin{table}
  \centering
  \caption{Theoretische Werte für die Landé-Faktoren der Orbitale.}
  \label{tab:theo1}
  \begin{tabular}{c c c c c c}
    \toprule
& Orbital  & S   &  L & J  & $g_j$ \\
    \midrule
Rot & $^1P_1$ & 0 & 1 & 1 & 1,0\\
&$^1D_2$& 0 & 2 & 2 & 1,0\\
Blau&$^3S_1$& 1 & 0 & 1 & 2,0\\
&$^3P_1$& 1 & 2 & 1 & 1,5\\
    \bottomrule
  \end{tabular}
\end{table}

Deweiteren könnnen die Lané-Faktoren für Übergänge
bei dem anormalen Zeeman-Effekt berechnet werden. Dafür werden nur
Übergänge zwischen Orbitale, die einen Spin $S=!0$, besitzen betrachtet (Blaues Licht).
Die Berechneten Werte sind in der Tabelle \ref{tab:theo2}
zu finden.

\begin{table}
  \centering
  \caption{Theoretische Werte für die Landé-Faktoren $g_{ji}$ für Übergänge bei
  dem der annormale Zeeman-Effekt berücksichtigt werden muss.}
  \label{tab:theo2}
\begin{tabular}{c c c c c c c}
  \toprule
      &            &  \multicolumn{2}{c}{$^3P_1$}  & \multicolumn{2}{c}{$^3P_1$} &    \\
      & $\Delta m$ &   $m_1$&  $m_1g_1$            & $m_2$    & $m_2g_2$         & $g_{ij}$\\
  \midrule
\sigma &  -1   &    1 &  2,0     &  0  & 0,0  &  2,0  \\
       &       &    0 &  0,0     & -1  & 1,5  &  1,5  \\
\pi    &   0   &    1 &  2,0     &  1  & 1,5  &  0,5  \\
       &       &    0 &  0,0     &  0  & 0,0  &  0,0  \\
       &       &   -1 & -2,0     & -1  & -1,5 & -0,5  \\
\sigma &  +1   &    0 &  0,0     &  1  & 1,5  & -1,5  \\
       &       &   -1 & -2,0     &  0  & 0,0  & -2,0  \\
\bottomrule
\end{tabular}
\end{table}

\subsection{Hysterese}
Es wurde eine Hysteresekurve für den verwendeten Elektromagneten aufgenommen.
Die Tabelle \ref{tab:hyst} enthält die Messwerte von des Magnetfeld $B$
in Abhängigkeit von dem  Strom $I$. Diese sind wiederum
in der Abblidung \ref{fig:hyst} aufgetragen.
\begin{table}
  \centering
  \caption{Messwerte für die Hysterese}
  \label{tab:hyst}
\begin{tabular}{c c c c}
  \toprule
 Strom $I/\si{\ampere}$ & Magnetfeld $B/\si{\milli\tesla}$  & $I/\si{\ampere}$ & $B/\si{\milli\tesla}$ \\
  \midrule
%messwerte
  \bottomrule
\end{tabular}
\end{table}


% \begin{figure}
%   \centering
%   \includegrafics{}
%   \caption
%   \label{fig:hyst}
% \end{figure}

Um Später die Benötigten B-Felder Zu bestimmen wird eine Ausgleichsrechnung
an den Messwerten bei steigendem Strom durch geführt.
Als Ansatz wird die Funktion
\begin{align}
B(I)=A \cdot \arctan(\frac{I-I_0}/C) \label{eqn:hyst}
\end{align}
verwendet.
Es ergeben sich für die Parameter die folgenden Werte:
\begin{align}
  A=\SI{1.39(7)e3}{\milli\tesla}   & I_0=\SI{0.21(13)}{\ampere}  &  C=\SI{21.6(16)}{ampere}
\end{align}
