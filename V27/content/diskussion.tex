\newpage
\section{Diskussion}
\label{sec:Diskussion}
Die gemessenen Landé-Faktoren für die unterschiedlich
Polarisationen und Wellenlängen sind noch einmal in der
Tabelle \ref{tab:ergebniss} aufgetragen
und werden mit dem theoretischen Berechneten verglichen.

Dabei muss beachtete werden, dass nur Positive $g_{ij}$
gemessen werden, da beim messen nicht zwischen rechs- und linkspolarisiertem Licht
unterschieden werden kann.
Folglich ist keine Aussage über das
Verhälniss von $\sigma_+$ und $\sigma_-$
möglich. Des weiteren
wird bei dem blauen \sigma-Übergang
nur der Mittelwert der beiden $g_{ij}$ gemessen ,
da die Auflösung nicht ausreicht zwischen beiden zu differenzieren.
Zum anderen ist bekannt, dass bei der blauen $\pi$-Übergang
$m_1=0$ zu $m_2=0$ stark unterdrückt ist. Folglich sollten nur die
anderen $\pi$-Übergänge beobachtet werden. Ebenfalls enthält
die Tabelle \ref{tab:ergebniss} die relativen
Abweichungen der Gemessenen- zu den Theoriewerten.

\begin{table}
  \centering
  \caption{Theoriewerte der Landé-Faktoren und die aus der Messung bestimmten Landé-Faktoren.}
  \label{tab:ergebniss}
  \begin{tabular}{c c c c c}
    \toprule
$ g_{\mathrm{ij}_\mathrm{theorie}} $ & $g_{\mathrm{ij}_\mathrm{gemessen}}$ & Abweichung\\
    \midrule
Rot  &$\sigma$ & 1    & 1,02\pm0,14 &0,02\pm0,14\\
\midrule
Blau &$\sigma$ & 1,75 & 1,88\pm0,26 &0,07\pm0,15\\
     &$\pi $   & 0,5  & 0,53\pm0,04 &0,06\pm0,08\\
    \bottomrule
  \end{tabular}
\end{table}

Aus der Tabelle \ref{tab:ergebniss} kann entnommen werden,
dass die Abweichungen zu dem Theoriewerten gering ausfällt.
Jedoch sind die Fehler auf den relativen Abweichungen
großer als die Abweichung selbst. Dies mag daran liegen, dass
zu wenig Ordnungen $n$ bei
der Bestimmung von $\Delta s$ und $\delta s$
beachtet worden sind.
Alles im allen eignet sich der Versuch zur Untersuchung
des Zeemann-Effektes da es möglich ist die Landé-Faktoren
für einen Übergang zu messen.
