\section{Diskussion}
\label{sec:Diskussion}
Die zu Beginn bestimmte vertikalkomponente des Erdmagnetfelds
\begin{align*}
B_{Erde_{z-gemessen}}&=\SI{33,7}{\micro\tesla}
\intertext{besitzt eine Abweichung  }
a&\approx0,25
\intertext{von dem Referenzwert}
B_{Erde_{z-ref}}&= \SI{45.2}{\micro\tesla} \text{\cite{wolf}}.
\end{align*}
Der Referenzwert ist nur die zeitlich gemittelte vertikalkomponente des Erdmagnetfeld
dieses kann jedoch für einen bestimmten Zeitraum variieren. Für eine genauere
Messung müssten mehrere Messungen über einen längeren durchgeführt werden.
Zusätzlich können andere ortliche Magnetfelder die stärke des zu messenden
Magnetfelders abschwächen was ebenfalls zu einer Abweichung führen kann.

Für die gemessenene Horizontalkomponente
\begin{align*}
B_{Erde_{horiz-gemessen}}&=\SI{20,2(16)}{\micro\tesla}
\intertext{beträgt die Abweichung}
a&=0\pm0,08
\intertext{für den Referenzwert}
B_{Erde_{horiz-ref}}&=20,191 \text{\cite{wolf}}.
\end{align*}
Diese geringe Abweichung spricht für eine genaue Messung, wobei
die Bemerkung für die Ungenauhigkeit für das Erdmagnetfeld
auch für diese Messung gilt.

Die aus den bestimmten Landé-Faktoren $g_F$ berechnete
Kernspins
\begin{align*}
I_{\ce{^{87}Rb}_{mess}}&=1,50\pm0,04     &\text{und}& &I_{\ce{^{85}Rb}_{mess}}&=2,52\pm0,06
\intertext{für die Isotope mit Kernspin}
I_{\ce{^{87}Rb}}&=\frac{3}{2} &\text{und}&  &I_{\ce{^{85}Rb}}&=\frac{5}{2} \text{\,\cite{verhalt}}
\intertext{besitzen eine Abweichung von}
a_{\ce{^{87}Rb}}&=0,003\pm0,023  &\text{und}&  &a_{\ce{^{85}Rb}}&=0,008\pm0,025.
\end{align*}
Die Abweichungen für die bestimmten Kernspins ist sehr gering. Somit
ist die Methode des Optischenpumpens sehr geeignet für die Bestimmung der
Energiedifferenzen der Energieniveaus, die durch den Zeemaneffekt und der
Hyperfeinstruktur entstehen, genau zu untersuchen.

Desweiteren konnte ein Isotopenverhältnissen von $\ce{^{87}Rb}$ und $\ce{^{85}Rb}$
bestimmt werden
\begin{align*}
  \frac{N_{\ce{^{87}Rb}}}{N_{\ce{^{85}Rb}}}&\approx 0,65
\intertext{dieses weicht von  dem Literaturwert }
 \frac{N_{\ce{^{87}Rb}}}{N_{\ce{^{85}Rb}}}_{Lit}&\approx 0,39 \text{\cite{verhalt}}
\intertext{um $a\approx0,67$ ab.}
\end{align*}
Diese große Abweichung lasst sich zum einen dadurch erklären,
dass die Methode zur Bestimmung der Amplituden zu ungenau ist.
Zum anderen könnte das
Verhältnis in der Probe ein anderes sein als das in der Natur.
Dies müsste man jedoch durch mehrere Messung genauer überprüfen.

% ??(Zu dem Quadratischen Zeemaneffekt kann gesagt werden, dass dieser
% kleiner ist folglich vernachlässigbar bei den verwendeten Magnetfelder)
