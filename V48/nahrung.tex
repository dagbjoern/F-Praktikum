\begin{longtable}{c c || c c}
  \caption{Aus den Messwerte berechneten Größen, die in dem Diagramm \ref{fig:logfit} aufgetragen sind.}
  \label{tab:nahrung}\\

  \toprule
  \multicolumn{2}{c}{1. Messung mit Heizrate $1,2 \si{\kelvin\per\minute}$ } & \multicolumn{2}{c}{2. Messung mit Heizrate $2,0 \si{\kelvin\per\minute}$ }\\
  $1/T \ \ \si{\per\kelvin} \cdot 10^{-3}$  & $\log(I)$ & $1/T \ \ \si{\per\kelvin} \cdot 10^{-3}$  & $\log(I)$\\
  \midrule
  \endfirsthead
  \hline
  \multicolumn{2}{c}{1. Messung mit Heizrate $1,2 \si{\kelvin\per\minute}$ } & \multicolumn{2}{c}{2. Messung mit Heizrate $2,0 \si{\kelvin\per\minute}$ }\\
  $1/T \ \ \si{\per\kelvin} \cdot 10^{-3}$  & $\log(I)$ & $1/T \ \ \si{\per\kelvin} \cdot 10^{-3}$  & $\log(I)$\\
  \midrule
  \endhead
  \hline
  \endfoot
  \bottomrule
  \endlastfoot

4,63 &   -26,5 &  4,66 &   -26,9 \\
4,56 &   -26,5 &  4,60  &   -26,4 \\
4,51 &   -26,5 &  4,58 &   -26,4 \\
4,46 &   -26,4 &  4,55 &   -26,4 \\
4,42 &   -26,2 &  4,52 &   -26,4 \\
4,39 &   -26   &  4,49 &   -26,2 \\
4,35 &   -25,6 &  4,46 &   -26,2 \\
4,32 &   -25,7 &  4,42 &   -26,1 \\
4,29 &   -25,6 &  4,38 &   -26,0   \\
4,27 &   -25,4 &  4,33 &   -25,8 \\
4,24 &   -25,3 &  4,29 &   -25,8 \\
4,22 &   -25,2 &  4,25 &   -25,8 \\
4,19 &   -25,1 &  4,21 &   -25,7 \\
4,17 &   -25   &  4,18 &   -25,5 \\
4,15 &   -24,8 &  4,16 &   -25,4 \\
4,13 &   -24,7 &  4,15 &   -25,3 \\
4,11 &   -24,6 &  4,13 &   -25,2 \\
4,09 &   -24,4 &  4,11 &   -25,0   \\
4,07 &   -24,3 &  4,10  &   -24,9 \\
4,05 &   -24,2 &  4,08 &   -24,7 \\
4,04 &   -24,1 &  4,07 &   -24,5 \\
4,02 &   -23,9 &  4,05 &   -24,4 \\
4,00   &   -23,8 &  4,04 &   -24,2 \\
3,99 &   -23,7 &  4,02 &   -24,1 \\
3,97 &   -23,5 &  4,01 &   -24,0   \\
3,96 &   -23,5 &  3,99 &   -23,8 \\
3,95 &   -23,4 &  3,98 &   -23,7 \\
3,94 &   -23,4 &  3,96 &   -23,5 \\
3,93 &   -23,3 &  3,95 &   -23,4 \\
3,93 &   -23,3 &  3,93 &   -23,2 \\
\end{longtable}
