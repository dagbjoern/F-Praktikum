\section{Diskussion}
\label{sec:Diskussion}
Die Bestimmung der Aktivierungsenergie  $W$ aus den zwei Messungen, mit
Hilfe der zwei Methoden \ref{sec:naherung} und \ref{sec:integral},
liefert die folgenden Ergebnisse:
\begin{align*}
\intertext{Für die Näherungs-Methode:}
W_1=(0,45\pm0,02)\si{\electronvolt}\\
W_2=(0,39\pm0,03)\si{\electronvolt}.
\intertext{Für die Integral-Methode:}
W_1=(0,64\pm0,02)\si{\electronvolt}\\
W_2=(0,62\pm0,02)\si{\electronvolt}.
\end{align*}
Dabei liefert die Integral-Methode bei beiden Messung ungfähr
denselben Wert, wohingegen die Werte
der Näherungs-Methode sich unterscheiden.
Da die Aktivierungsenergie für belibige Heizrate immer konstant ist,
lässt sich die Behauptung aus \cite{skript},
dass sich $W$ genauer aus dem gesamten Kurvenverlauf ermitteln lässt,
aus den Ergebnissen bestätigen.
Desweiteren wird aus den bestimmten Aktivierungsenergien $W$
die charakteristische Relaxationszeit $\tau_0$ berechnet \ref{sec:relax}.
Die Ergebnisse für $\tau_0$ sind:
\begin{align*}
\tau_{0_1}^\mathrm{Näherung}&=(9\pm7)\cdot10^{-7}\si{\second}\\
\tau_{0_2}^\mathrm{Näherung}&=(1.6\pm2.0)\cdot10^{-5}\si{\second}\\
\tau_{0_1}^\mathrm{Integral}&=(1.7\pm1.3)\cdot10^{-10}\si{\second}\\
\tau_{0_2}^\mathrm{Integral}&=(4\pm4)\cdot10^{-10}\si{\second}.
\end{align*}
Diese Ergebnisse sind jedoch nicht aussagekräftig,
da die Fehler für $tau_0$ in derselben Größenordnung wie die bestimmten Werte liegen und
teilweise größer sind als
die Werte selbst.
Diese Fehler sind der Ungenauigkeit der Messung geschuldet, da
zu einem nicht immer die Heizrate konstant gehalten werden konnte,
weil diese nur per Hand geregelt wurde, und zum anderen
das verwendete Amperemeter eventuell zu ungenau war.
