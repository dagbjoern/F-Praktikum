\section{Diskussion}
\label{sec:Diskussion}
Die Bestimmung der Aktivierungsenergie  $W$ aus den zwei Messungen mit
Hilfe der zwei Methoden \ref{sec:naherung} und \ref{sec:integral}
lieferte die folgenden Ergebnisse:
\begin{align*}
\intertext{Für die Näherungs-Methode:}
W_1=(0,45\pm0,02)\si{\electronvolt}\\
W_2=(0,39\pm0,03)\si{\electronvolt}
\intertext{Für die Integral-Methode}
W_1=(0,64\pm0,02)\si{\electronvolt}\\
W_2=(0,62\pm0,02)\si{\electronvolt}
\end{align*}
Dabei liefert die Integral-Methode bei beiden Messung ungfähr
denselben Wert wohingegen die Werte
der Näherungs-Methode sich unterscheiden.
Da die Aktivierungsenergie egal für welche Heizrate immer konstant ist,
lässt sich die Behauptung aus \cite{skript},
das sich $W$ genauer aus dem gesamten Kurvenverlauf ermitteln lässt,
Ergebnissen bestätigen.
Desweiteren wurden aus denn gemessenen Aktivierungsenergie $W$
die charakteristische Relaxationszeit $\tau_0$ bestimmt \ref{sec:relax}.
Die Ergebnisse für $\tau_0$, die aus den unterschiedlich
bestimmten Aktivierungsenergien berechnet wurden, sind:
\begin{align*}
\tau_{0_1}^\mathrm{Näherung}&=(9\pm7)\cdot10^{-7}\si{\second}\\
\tau_0^\mathrm{Näherung \ 2}&=(1.6\pm2.0)\cdot10^{-5}\si{\second}\\
\tau_0^\mathrm{Integral \ 1}&=(1.7\pm1.3)\cdot10^{-10}\si{\second}\\
\tau_0^\mathrm{Integral \ 2}&=(4\pm4)\cdot10^{-10}\si{\second}
\end{align*}
Diese Ergebnisse sind jedoch nicht aussagekräftig,
da der Fehler der
Ergebnisse für $\tau_0$ in derselben Größenordnung liegen und
teilweise größer ist als
der Wert selbst.
Diese Fehler sind der Ungenauigkeit der Messung geschuldet, da
zu einem nicht immer die Heizrate konstant gehalten werden konnte,
weil diese nur per Hand geregelt wurde, und zum anderen
das verwendete Amperemeter vielleicht zu ungenau war.
