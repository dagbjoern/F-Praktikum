\section{Diskussion}
\label{sec:Diskussion}
Die Bestimmung der Aktivierungsenergie  $W$ aus den zwei Messungen, mit
Hilfe der zwei Methoden \ref{sec:naherung} und \ref{sec:integral},
liefert die folgenden Ergebnisse mit den Abweichungen
zu dem Literaturwert:
 \begin{align*}
 W_\mathrm{Lit}=0,66\ \si{\electronvolt}\text{\cite{litref}}.
 \end{align*}
\begin{align*}
\intertext{Näherungs-Methode:}
W_1&=(0,610\pm0,006)\ \si{\electronvolt}&   &a=0,075\pm0,009 \\
W_2&=(0,812\pm0,017)\ \si{\electronvolt}&   &a=0,230\pm0,026.
\intertext{Integral-Methode:}
W_1&=(0,782\pm0,008)\ \si{\electronvolt}&  &a=0,185\pm0,012\\
W_2&=(0,841\pm0,019)\ \si{\electronvolt}&  &a=0,275\pm0,028.
\end{align*}

Dabei liefert die Nährungs-Methode bei beiden Messung einen genaueren Wert
als die Integral-Methode. Folglich
lässt sich die Behauptung aus \cite{skript},
dass sich $W$ genauer aus dem gesamten Kurvenverlauf ermitteln lässt,
aus den Ergebnissen nicht bestätigen.
Desweiteren wird aus den bestimmten Aktivierungsenergien $W$
die charakteristische Relaxationszeit $\tau_0$ berechnet \ref{sec:relax}.
Der Literaturwert für $\tau_0$ ist
\begin{align*}
\tau_{\mathrm{Lit}}=4\cdot10^{-14} \si{\second} \text{\cite{litref}}.
\end{align*}
\begin{align*}
  \intertext{Die Ergebnisse und Abweichungen für $\tau_0$ sind:}
  \tau_{0_1}^\mathrm{Näherung}&=(5,6\pm1,7)\cdot10^{-10}\ \si{\second}&  &a=(1,4\pm0,4)\cdot10^4\\
  \tau_{0_2}^\mathrm{Näherung}&=(6,0\pm5,0)\cdot10^{-14}\ \si{\second}&  &a=3,9\pm11,9\\
  \tau_{0_1}^\mathrm{Integral}&=(1,9\pm0,7)\cdot10^{-13}\ \si{\second}&  &a=0,5\pm1,2\\
  \tau_{0_2}^\mathrm{Integral}&=(1,6\pm1,4)\cdot10^{-14}\ \si{\second}&  &a=0,60\pm0,34.
\end{align*}
Diese Ergebnisse sind jedoch nicht aussagekräftig,
da die Fehler für $\tau_0$ in derselben Größenordnung wie die bestimmten Werte liegen und die
Abweichungen zum Literaturwerte zu groß sind.
