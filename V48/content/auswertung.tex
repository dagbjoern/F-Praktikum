\section{Auswertung}
\label{sec:Auswertung}

\subsection{Untergrund}
\label{sec:Untergrund}

Um die Aktivierungsenergie $W$ von der Probe zu bestimmen,
müssen zu nächst die gemessenen
Messwerte, die in der Tabelle \ref{tab:messwerte} aufgelistet sind,
für die unterschiedlichen Heizraten in der Form
Temperatur gegen Depolarisationsstrom aufgetragen werden, zu sehen
in der Abbildung \ref{fig:mess}.

% \begin{table}
\begin{longtable}{c c ||c c}
  \caption{Messwerte für die unterschiedlichen Heizraten.}
 \label{tab:messwerte}\\
 \toprule
\multicolumn{2}{c}{1. Messung mit Heizrate $1,2 \si{\kelvin\per\minute}$ }  &  \multicolumn{2}{c}{2. Messung mit Heizrate $2,0 \si{\kelvin\per\minute}$ }\\
   Temperatur $T_1 \si{\kelvin}$ &   Strom $I_1 \si{\pico\ampere}$ &   Temperatur $T_2 \si{\kelvin}$ &   Strom $I_2 \si{\pico\ampere}$ \\
\midrule
\endfirsthead
\hline
\multicolumn{2}{c}{1. Messung mit Heizrate $1,2 \si{\kelvin\per\minute}$ }  &  \multicolumn{2}{c}{2. Messung mit Heizrate $2,0 \si{\kelvin\per\minute}$ }\\
   Temperatur $T_1 \si{\kelvin}$ &   Strom $I_1 \si{\pico\ampere}$ &   Temperatur $T_2 \si{\kelvin}$ &   Strom $I_2 \si{\pico\ampere}$ \\
\midrule
\endhead
\hline
\endfoot
\bottomrule
\endlastfoot

216,15 &        3   &          214,65 &        2   \\
219,25 &        3   &          217,25 &        3,5 \\
221,55 &        3   &          218,45 &        3,5 \\
224,25 &        3,5 &          219,85 &        3,5 \\
226,35 &        4   &          221,25 &        3,5 \\
228,05 &        5   &          222,65 &        4   \\
229,65 &        8   &          224,45 &        4   \\
231,25 &        7   &          226,45 &        4,5 \\
232,85 &        8   &          228,55 &        5   \\
234,25 &        9   &          230,75 &        6   \\
235,85 &       10   &          233,15 &        7   \\
237,05 &       11   &          235,25 &        9   \\
238,55 &       12   &          237,25 &       11,5 \\
239,55 &       14   &          239,25 &       15   \\
240,85 &       17   &          240,25 &       17   \\
242,05 &       19   &          241,25 &       19   \\
243,25 &       22   &          242,25 &       21   \\
244,45 &       26   &          243,15 &       25   \\
245,55 &       30,5 &          244,15 &       28   \\
246,65 &       35   &          245,05 &       32   \\
247,65 &       38   &          245,95 &       36   \\
248,75 &       47   &          246,85 &       40,5 \\
249,75 &       52   &          247,75 &       46   \\
250,75 &       59   &          248,65 &       52   \\
251,95 &       67   &          249,55 &       57   \\
252,35 &       71   &          250,55 &       65   \\
252,95 &       75   &          251,35 &       73   \\
253,55 &       78   &          252,25 &       81   \\
254,15 &       84   &          253,25 &       91   \\
254,75 &       88   &          254,25 &      105   \\
255,25 &       94   &          255,05 &      115   \\
255,85 &      100   &          256,05 &      125   \\
256,55 &      100   &          257,05 &      135   \\
257,95 &      105   &          257,95 &      150   \\
258,55 &      110   &          258,95 &      160   \\
259,15 &      110   &          259,95 &      165   \\
259,55 &      110   &          260,95 &      170   \\
260,25 &      110   &          262,05 &      175   \\
261,15 &      100   &          263,05 &      175   \\
261,75 &       95   &          264,05 &      175   \\
262,15 &       90   &          265,15 &      170   \\
262,75 &       90   &          266,15 &      160   \\
263,25 &       81   &          267,25 &      145   \\
263,85 &       77   &          268,25 &      130   \\
264,35 &       73   &          269,25 &      115   \\
264,95 &       67   &          270,25 &      100   \\
265,45 &       62   &          271,25 &       85   \\
266,05 &       57   &          272,25 &       75   \\
267,25 &       48   &          273,25 &       60   \\
268,35 &       40   &          274,15 &       55   \\
269,55 &       30   &          276,05 &       49   \\
270,75 &       28   &          277,05 &       52   \\
271,85 &       25   &          278,05 &       51   \\
272,75 &       24   &          279,05 &       51   \\
273,65 &       24   &          279,85 &       51   \\
274,85 &       24   &          280,85 &       51   \\
276,05 &       25   &          281,85 &       51   \\
277,25 &       25   &          282,75 &       53   \\
278,35 &       26   &          283,65 &       54   \\
279,35 &       27   &          285,65 &       56   \\
280,45 &       27   &          287,45 &       57   \\
281,65 &       28   &          289,55 &       60   \\
282,85 &       29,5 &          291,45 &       62   \\
284,05 &       30,5 &                 &            \\
285,25 &       31,5 &                 &            \\
286,45 &       32,5 &                 &            \\
287,65 &       33,5 &                 &            \\
288,75 &       34   &                 &            \\
289,85 &       34,5 &                 &            \\
291,05 &       35   &                 &            \\
\end{longtable}


\begin{figure}
  \centering
  \begin{subfigure}{0.49\textwidth}
    \centering
    \includegraphics[width=1.0\textwidth]{build/plot1_messwerte}
    \caption{1. Messung mit der Heizrate $1,2 \si{\kelvin\per\minute}$}
    \label{fig:mess1}
  \end{subfigure}
  \begin{subfigure}{0.49\textwidth}
    \centering
    \includegraphics[width=1.0\textwidth]{build/plot2_messwerte}
    \caption{2. Messung mit der Heizrate $2,0 \si{\kelvin\per\minute}$.}
    \label{fig:mess2}
  \end{subfigure}
\caption{Messwerte der beiden Messungen mit unterschiedlichen Heizraten.}
\label{fig:mess}
\end{figure}


In der Abbildung \ref{fig:mess} wurden bestimmte
Messpunkte mit der Farbe Blau
markiert, diese dienen zum Herausrechen des Untergrundes
bei den Messungen indem eine lineare Regression auf diese Messwerten
durchgeführt wird.
\begin{align*}
m_1&= (7,3\pm0,2)\cdot10^{-13}      &b_1 =(-1.77\pm0,06)\cdot10^{-10} \\
m_2&=(10,5\pm0,4)\cdot10^{-13}      &b_2 =(-2.45\pm0,14)\cdot10^{-10}
\end{align*}

Der so berechnete Untergrund
\begin{align*}
  I_\mathrm{Untergrund}=(m_\mathrm{reg}\cdot T_\mathrm{mess}+b_\mathrm{reg})
\end{align*}
wird nun von den Messungen
abgezogen, jedoch nur solange der berechnete Wert
des Untergrundes positiv ist.
\begin{equation}
 I_{\mathrm{ohne}}=I_{\mathrm{mess}}-I_\mathrm{Untergrund}\label{eqn:Untergrund}
\end{equation}
In der Abbildung \ref{fig:korr} sind die Ausgleichsgerade des Untergrundes
sowie die Messwerte ohne Untergrund aufgetragen.

\begin{figure}
  \centering
  \begin{subfigure}{0.49\textwidth}
    \centering
    \includegraphics[width=1.0\textwidth]{build/plot1_korrektur}
    \caption{1. Messung mit der Heizrate $1,2 \si{\kelvin\per\minute}$}
    \label{fig:korr1}
  \end{subfigure}
  \begin{subfigure}{0.49\textwidth}
    \centering
    \includegraphics[width=1.0\textwidth]{build/plot2_korrektur}
    \caption{2. Messung mit der Heizrate $2,0 \si{\kelvin\per\minute}$.}
    \label{fig:korr2}
  \end{subfigure}
\caption{Messwerte sowie die Messwerte ohne Untergrund  und der berechnete Untergrund der beiden Messungen mit unterschiedlichen Heizraten.}
\label{fig:korr}
\end{figure}


\subsection{Aktivierungsenergie}
\label{sec:Aktivierungsenergie}
\subsubsection{Näherungs-Methode}
\label{sec:naherung}
Nun kann die Aktivierungsenergie $W$ berechnet werden.
Bei der ersten Methode wird davon ausgegangen, dass am Anfangsteil
der Depolarisationskurve das Integral in der Gleichung \eqref{eqn:8}
ungefähr gleich Null ist folglich gilt dort die Näherung
\begin{equation}
j(T)=\frac{p^2EN_\mathrm{p}}{3kT_\mathrm{p}\tau_{0}}\exp\left(\frac{-W}{kt}\right) \label{eqn:9}
\end{equation}
Die Messwerte, für die diese Näherung gilt ,sind
in den Abbildung \ref{fig:fit} Grün markiert
und werden in der Form
$\ln I$ gegen $\frac{1}{T}$ in dem Diagramm
\ref{fig:logfit} aufgetragen und in sind ebenfalls in der Tabelle \ref{tab:nahrung} zu finden.

% tabellle
\begin{longtable}{c c || c c}
  \caption{Aus den Messwerte berechneten Größen, die in dem Diagramm \ref{fig:logfit} aufgetragen sind.}
  \label{tab:nahrung}\\

  \toprule
  \multicolumn{2}{c}{1. Messung mit Heizrate $1,2 \si{\kelvin\per\minute}$ } & \multicolumn{2}{c}{2. Messung mit Heizrate $2,0 \si{\kelvin\per\minute}$ }\\
  $1/T \ \ \si{\per\kelvin} \cdot 10^{-3}$  & $\log(I)$ & $1/T \ \ \si{\per\kelvin} \cdot 10^{-3}$  & $\log(I)$\\
  \midrule
  \endfirsthead
  \hline
  \multicolumn{2}{c}{1. Messung mit Heizrate $1,2 \si{\kelvin\per\minute}$ } & \multicolumn{2}{c}{2. Messung mit Heizrate $2,0 \si{\kelvin\per\minute}$ }\\
  $1/T \ \ \si{\per\kelvin} \cdot 10^{-3}$  & $\log(I)$ & $1/T \ \ \si{\per\kelvin} \cdot 10^{-3}$  & $\log(I)$\\
  \midrule
  \endhead
  \hline
  \endfoot
  \bottomrule
  \endlastfoot

4,63 &   -26,5 &  4,66 &   -26,9 \\
4,56 &   -26,5 &  4,60  &   -26,4 \\
4,51 &   -26,5 &  4,58 &   -26,4 \\
4,46 &   -26,4 &  4,55 &   -26,4 \\
4,42 &   -26,2 &  4,52 &   -26,4 \\
4,39 &   -26   &  4,49 &   -26,2 \\
4,35 &   -25,6 &  4,46 &   -26,2 \\
4,32 &   -25,7 &  4,42 &   -26,1 \\
4,29 &   -25,6 &  4,38 &   -26,0   \\
4,27 &   -25,4 &  4,33 &   -25,8 \\
4,24 &   -25,3 &  4,29 &   -25,8 \\
4,22 &   -25,2 &  4,25 &   -25,8 \\
4,19 &   -25,1 &  4,21 &   -25,7 \\
4,17 &   -25   &  4,18 &   -25,5 \\
4,15 &   -24,8 &  4,16 &   -25,4 \\
4,13 &   -24,7 &  4,15 &   -25,3 \\
4,11 &   -24,6 &  4,13 &   -25,2 \\
4,09 &   -24,4 &  4,11 &   -25,0   \\
4,07 &   -24,3 &  4,10  &   -24,9 \\
4,05 &   -24,2 &  4,08 &   -24,7 \\
4,04 &   -24,1 &  4,07 &   -24,5 \\
4,02 &   -23,9 &  4,05 &   -24,4 \\
4,00   &   -23,8 &  4,04 &   -24,2 \\
3,99 &   -23,7 &  4,02 &   -24,1 \\
3,97 &   -23,5 &  4,01 &   -24,0   \\
3,96 &   -23,5 &  3,99 &   -23,8 \\
3,95 &   -23,4 &  3,98 &   -23,7 \\
3,94 &   -23,4 &  3,96 &   -23,5 \\
3,93 &   -23,3 &  3,95 &   -23,4 \\
3,93 &   -23,3 &  3,93 &   -23,2 \\
\end{longtable}



\begin{figure}
  \centering
  \begin{subfigure}{0.49\textwidth}
    \centering
    \includegraphics[width=1.0\textwidth]{build/plot1_fit}
    \caption{1. Messung mit der Heizrate $1,2 \si{\kelvin\per\minute}$}
    \label{fig:fit1}
  \end{subfigure}
  \begin{subfigure}{0.49\textwidth}
    \centering
    \includegraphics[width=1.0\textwidth]{build/plot2_fit}
    \caption{2. Messung mit der Heizrate $2,0 \si{\kelvin\per\minute}$.}
    \label{fig:fit2}
  \end{subfigure}
\caption{Messwerte ohne Untergrund mit markierten Werten für welche
die Näherung \ref{eqn:9} gilt.}
\label{fig:fit}
\end{figure}





\begin{figure}
  \centering
  \begin{subfigure}{0.49\textwidth}
    \centering
    \includegraphics[width=1.0\textwidth]{build/plot3_messung1}
    \caption{1. Messung mit der Heizrate $1,2 \si{\kelvin\per\minute}$}
    \label{fig:logfit1}
  \end{subfigure}
  \begin{subfigure}{0.49\textwidth}
    \centering
    \includegraphics[width=1.0\textwidth]{build/plot3_messung2}
    \caption{2. Messung mit der Heizrate $2,0 \si{\kelvin\per\minute}$.}
    \label{fig:logfit2}
  \end{subfigure}
\caption{Messwerte für welche
die Näherung \ref{eqn:9} gilt und lineare Regression durch eben diese.}
\label{fig:logfit}
\end{figure}


Durch eine lineare Regression
mit den Werten, die in der Abbildung \ref{fig:logfit} dargestellt sind, können
so aus der umgestellten Formel \ref{eqn:9}  $m$ und $b$ aus \eqref{eqn:mb} berechnet werden.
\begin{align}
  \underbrace{\ln(I(1/T))}_{y(x)}\approx \underbrace{-\frac{W}{k}}_{m}\cdot \underbrace{\frac1T}_{x}+\underbrace{\ln \left( \frac{p^2E}{3kT_p} \frac{N_p}{\tau_0}\right)}_{b} \label{eqn:mb}
\end{align}
Die Aktivierungsenergie $W$ kann somit über
\begin{align}
  W=-k\cdot m \label{eqn:Wcool}
\end{align}
berechnet werden.
Aus den linearen Regressionen folgt:
\begin{align*}
  m_1=-(5,29\pm0,19)\cdot10^{3}    &  &b_1=-(0,0\pm1,9)\cdot10^{2}\\
  m_2=-(4,52\pm0,31)\cdot10^{3}    &  &b_2=-(0,1\pm3,1)\cdot10^{2}
\end{align*}
Somit ergibt sich aus der Formel \eqref{eqn:Wcool} die Aktivierungsenergien:
\begin{align*}
  W_1=(0,45\pm0,02)\si{\electronvolt}\\
  W_2=(0,39\pm0,03)\si{\electronvolt}
\end{align*}

\subsubsection{Integral-Methode}
\label{sec:integral}
Die Aktivierungsenergie kann ebenfalls durch den Zusammenhang \eqref{eqn:int}
berechnet werden.

\begin{align}
  \frac{W}{kT}+\ln{(const)}=\ln \frac{\int_T^{T^*}{I(T')dT'}}{I(T)} \label{eqn:int}
\end{align}

Wieder wird $1/T$ nun aber gegen
\begin{align}
  \ln \int_T^{T^*}I(T')dT'\frac{1}{I(T)}=B(\frac{1}{T}) \label{eqn:intcool}
\end{align}
aufgetragen und eine
lineare Regression auf den Messwerten,
die in der Tabelle \ref{tab:integral} aufgelistet sind,
durchgeführt,
zu sehen in der Abbildung \ref{fig:intreg}.
Die lineare Regression vernachlässigt die Wertpaare,
die zum Berechnen der Untergrundes genutzt wurden,
da diese zu starke Schwankungen aufweisen.
Dabei ist $T^*$ ein feste Temperatur mit $I(T^*)\approx 0$.
Das Integral in der Formel \ref{eqn:intcool} von $T$ bis $T^*$
wird mit der Funktion trapz aus der Python Bibliothek scipy.integrate
berechnete.

\begin{longtable}{c c || c c}
  \caption{Aus den Messwerte berechneten Größen die in dem Diagramm \ref{fig:intreg} aufgetragen sind,}
  \label{tab:integral}\\
  \toprule
  \multicolumn{2}{c}{1. Messung mit Heizrate $1,2 \si{\kelvin\per\minute}$ } & \multicolumn{2}{c}{2. Messung mit Heizrate $2,0 \si{\kelvin\per\minute}$ }\\
  $1/T \ \ \si{\per\kelvin} \cdot 10^{-3}$  & $\log(I)$ & $1/T \ \ \si{\per\kelvin} \cdot 10^{-3}$  & $\log(I)$\\
  \midrule
  \endfirsthead
  \hline
  \multicolumn{2}{c}{1. Messung mit Heizrate $1,2 \si{\kelvin\per\minute}$ } & \multicolumn{2}{c}{2. Messung mit Heizrate $2,0 \si{\kelvin\per\minute}$ }\\
  $1/T \ \ \si{\per\kelvin} \cdot 10^{-3}$  & $B(\frac{1}{T})$ & $1/T \ \ \si{\per\kelvin} \cdot 10^{-3}$  & $B(\frac{1}{T})$\\
  \midrule
  \endhead
  \hline
  \endfoot
  \bottomrule
  \endlastfoot
4,63 &       6,37 &  4,66 &        7,18 \\
4,56 &       6,36 &  4,60  &        6,62 \\
4,51 &       6,36 &  4,58 &        6,62 \\
4,46 &       6,20  &  4,55 &        6,62 \\
4,42 &       6,06 &  4,52 &        6,62 \\
4,39 &       5,83 &  4,49 &        6,48 \\
4,35 &       5,36 &  4,46 &        6,48 \\
4,32 &       5,48 &  4,42 &        6,36 \\
4,29 &       5,34 &  4,38 &        6,25 \\
4,27 &       5,22 &  4,33 &        6,06 \\
4,24 &       5,10  &  4,29 &        5,97 \\
4,22 &       5,00    &  4,25 &        6,00    \\
4,19 &       4,90  &  4,21 &        5,93 \\
4,17 &       4,74 &  4,18 &        5,74 \\
4,15 &       4,53 &  4,16 &        5,62 \\
4,13 &       4,41 &  4,15 &        5,52 \\
4,11 &       4,27 &  4,13 &        5,42 \\
4,09 &       4,12 &  4,11 &        5,17 \\
4,07 &       3,96 &  4,10  &        5,04 \\
4,05 &       3,81 &  4,08 &        4,86 \\
4,04 &       3,72 &  4,07 &        4,70  \\
4,02 &       3,47 &  4,05 &        4,54 \\
4,00   &      3,35 &  4,04 &        4,37 \\
3,99 &       3,18 &  4,02 &        4,20  \\
3,97 &       3,00    &  4,01 &      4,08 \\
3,96 &       2,92 &  3,99 &        3,9  \\
3,95 &       2,83 &  3,98 &        3,73 \\
3,94 &       2,76 &  3,96 &        3,59 \\
3,93 &       2,64 &  3,95 &        3,42 \\
3,93 &       2,55 &  3,93 &        3,21 \\
3,92 &       2,44 &  3,92 &        3,07 \\
3,91 &       2,32 &  3,91 &        2,93 \\
3,90  &       2,26 &  3,89 &        2,78 \\
3,88 &       2,05 &  3,88 &        2,6  \\
3,87 &       1,92 &  3,86 &        2,45 \\
3,86 &       1,83 &  3,85 &        2,33 \\
3,85 &       1,77 &  3,83 &        2,2  \\
3,84 &       1,65 &  3,82 &        2,04 \\
3,83 &       1,59 &  3,80  &        1,91 \\
3,82 &       1,52 &  3,79 &        1,76 \\
3,81 &       1,50  &  3,77 &        1,59 \\
3,81 &       1,37 &  3,76 &        1,46 \\
3,80  &       1,37 &  3,74 &        1,33 \\
3,79 &       1,28 &  3,73 &        1,21 \\
3,78 &       1,21 &  3,71 &        1,08 \\
3,77 &       1,15 &  3,70  &        0,95 \\
3,77 &       1,10  &  3,69 &        0,85 \\
3,76 &       1,02 &  3,67 &        0,67 \\
3,74 &       0,83 &  3,66 &        0,84 \\
3,73 &       0,65 &  3,65 &        0,88 \\
3,71 &       0,74 &  3,62 &        1,38 \\
3,69 &       0,40  &  3,61 &        0,52 \\
3,68 &       0,33 &  3,60  &        0,37 \\
3,67 &       0,27 &  3,58 &       -0,11 \\
3,65 &      -0,16 &  3,57 &       -0,90  \\
\end{longtable}




\begin{figure}
  \centering
  \begin{subfigure}{0.49\textwidth}
    \centering
    \includegraphics[width=1.0\textwidth]{build/plot5_1}
    \caption{1. Messung mit der Heizrate $1,2 \si{\kelvin\per\minute}$}
    \label{fig:intreg1}
  \end{subfigure}
  \begin{subfigure}{0.49\textwidth}
    \centering
    \includegraphics[width=1.0\textwidth]{build/plot5_2}
    \caption{2. Messung mit der Heizrate $2,0 \si{\kelvin\per\minute}$.}
    \label{fig:intreg2}
  \end{subfigure}
\caption{Lineare Regression um $W$ aus dem gesamten Kurvenverlauf zu bestimmen.}
\label{fig:intreg}
\end{figure}

Aus den linearen Regressionen folgt diesmal:
\begin{align*}
  m_1=(7,44\pm0,19)\cdot10^{3}    &  &b_1=-(0,3\pm1,9)\cdot10^{2}\\
  m_2=(7,18\pm0.27)\cdot10^{3}    &  &b_2=-(0.3\pm3,1)\cdot10^{2}
\end{align*}
Somit ergibt sich aus der Formel \eqref{eqn:int} die Aktivierungsenergien:
\begin{align*}
  W_1=(0,64\pm0,02)\si{\electronvolt}\\
  W_2=(0,62\pm0,02)\si{\electronvolt}
\end{align*}


\subsection{Charakteristische Relaxationszeit}
\label{sec:relax}
Durch Differentiation der Gleichung \eqref{eqn:8}
ergibt sich der Zusammenhang \eqref{eqn:diff}
zwischen $T_\mathrm{max}$ und der charakteristischen Relaxationszeit $\tau_0$.
\begin{align}
  \tau_0=\frac{k T_\mathrm{max}^2}{b W} \exp\left(-\frac{W}{k T_\mathrm{max}} \right)  \label{eqn:diff}
\end{align}
Somit ergibt sich für die erste Messung mit der Heizrate
\begin{align*}
b&=1,2 \si{\kelvin\per\minute}
\intertext{sowie der aus den Messwerten bestimmten Größe}
T_\mathrm{max}&=(260\pm1)\si{\kelvin}
\intertext{und den zwei unterschiedlich berechneten
Aktivierungsenergien}
\tau_0^\mathrm{Näherung \ 1}&=(9\pm7)\cdot10^{-7}\si{\second},\\
\tau_0^\mathrm{Integral \ 1}&=(1.7\pm1.3)\cdot10^{-10}\si{\second}.
\end{align*}
Für die zweite Messung mit
\begin{align*}
b&=2 \si{\kelvin\per\minute}
\intertext{und}
T_\mathrm{max}&=(263\pm1)\si{\kelvin}
\intertext{folgt }
\tau_0^\mathrm{Näherung \ 2}&=(1.6\pm2,0)\cdot10^{-5}\si{\second},\\
\tau_0^\mathrm{Integral \ 2}&=(4\pm4)\cdot10^{-10}\si{\second}.
\end{align*}
