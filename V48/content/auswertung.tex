\section{Auswertung}
\label{sec:Auswertung}

\subsection{Untergrund}
\label{sec:Untergrund}

Um die Aktivierungsenergie $W$ von der Probe zu bestimmen,
müssen zu nächst die gemessenen
Messwerte für die unterschiedlichen Heizraten in der Form
Temperatur gegen Depolarisationsstrom aufgetragen werden, zu sehen
in der Abbildung \ref{fig:mess}.

\begin{figure}
  \centering
  \begin{subfigure}{0.48\textwidth}
    \centering
    \includegraphics[height=5.0cm]{build/plot1_messwerte}
    \caption{Messung mit der Heizrate $1,2 \si{\kelvin\per\minute}$}
    \label{fig:mess1}
  \end{subfigure}
  \begin{subfigure}{0.48\textwidth}
    \centering
    \includegraphics[height=5.0cm]{build/plot2_messwerte}
    \caption{Messung mit der Heizrate $2 \si{\kelvin\per\minute}$.}
    \label{fig:mess2}
  \end{subfigure}
\caption{Messwerte der beiden Messungen mit unterschiedlichen Heizraten.}
\label{fig:mess}
\end{figure}


In der Abbildung \ref{fig:mess} wurden bestimmte
Messpunkte mit der Farbe Blau
makiert, diese dienen zum Herausrechen des Untergrundes
bei den Messungen indem eine lineare Regression auf diese Messwerten
durchgeführt wird.
\begin{align*}
m_1&= (7,3\pm0,2)\cdot10^{-13}      &b_1 =(-1.77\pm0,06)\cdot10^{-10} \\
m_2&=(10,5\pm0,4)\cdot10^{-13}      &b_2 =(-2.45\pm0,14)\cdot10^{-10}
\end{align*}

Der so berechnete Untergrund
\begin{align*}
  I_\mathrm{Untergrund}=(m_\mathrm{reg}\cdot T_\mathrm{mess}+b_\mathrm{reg})
\end{align*}
wird nun von den Messungen
abgezogen, jedoch nur solange der berechnete Wert
des Untergrundes positiv ist.
\begin{equation}
 I_{\mathrm{ohne}}=I_{\mathrm{mess}}-I_\mathrm{Untergrund}\label{eqn:Untergrund}
\end{equation}
In der Abbildung \ref{fig:korr} sind die Ausgleichsgerade des Untergrundes
sowie die Messwerte ohne Untergrund aufgetragen.

\begin{figure}
  \centering
  \begin{subfigure}{0.48\textwidth}
    \centering
    \includegraphics[height=5.0cm]{build/plot1_korrektur}
    \caption{1.Messung mit der Heizrate $1,2 \si{\kelvin\per\minute}$}
    \label{fig:korr1}
  \end{subfigure}
  \begin{subfigure}{0.48\textwidth}
    \centering
    \includegraphics[height=5.0cm]{build/plot2_korrektur}
    \caption{2.Messung mit der Heizrate $2 \si{\kelvin\per\minute}$.}
    \label{fig:korr2}
  \end{subfigure}
\caption{Messwerte sowie die Messwerte ohne Untergrund  und der berechnete Untergrund der beiden Messungen mit unterschiedlichen Heizraten.}
\label{fig:korr}
\end{figure}

\subsection{Aktivierungsenergie}
\label{sec:Aktivierungsenergie}
Nun kann die Aktivierungsenergie $W$ berechnet werden, indem die Messwerte,
die Grün in den Abbildung \ref{fig:fit} makiert sind, in der Form
$\ln I$ gegen $\frac{1}{T}$ in dem Diagramm
\ref{fig:logfit} aufgetragen werden, da für diese die Näherung
\ref{eqn:??} gilt.


\begin{figure}
  \centering
  \begin{subfigure}{0.48\textwidth}
    \centering
    \includegraphics[height=5.0cm]{build/plot1_fit}
    \caption{1.Messung mit der Heizrate $1,2 \si{\kelvin\per\minute}$}
    \label{fig:fit1}
  \end{subfigure}
  \begin{subfigure}{0.48\textwidth}
    \centering
    \includegraphics[height=5.0cm]{build/plot2_fit}
    \caption{2.Messung mit der Heizrate $2 \si{\kelvin\per\minute}$.}
    \label{fig:fit2}
  \end{subfigure}
\caption{Messwerte ohne Untergrund mit markierten Werten für die
die Näherung \ref{eqn:??} gilt.}
\label{fig:fit}
\end{figure}





\begin{figure}
  \centering
  \begin{subfigure}{0.48\textwidth}
    \centering
    \includegraphics[height=5.0cm]{build/plot3_messung1}
    \caption{1.Messung mit der Heizrate $1,2 \si{\kelvin\per\minute}$}
    \label{fig:logfit1}
  \end{subfigure}
  \begin{subfigure}{0.48\textwidth}
    \centering
    \includegraphics[height=5.0cm]{build/plot3_messung2}
    \caption{2.Messung mit der Heizrate $2 \si{\kelvin\per\minute}$.}
    \label{fig:logfit2}
  \end{subfigure}
\caption{Messwerte für die
die Näherung \ref{eqn:??} gilt und linerare Regression durch eben diese.}
\label{fig:logfit}
\end{figure}


Durch eine lineare Regression
mit den Werte, die in der Abbildung \ref{logfig} dargestellt sind, können
so durch die Formel \ref{eqn:??} $m$ und $b$ berechnet werden.
\begin{align}
  \underbrace{\ln(I(1/T))}_{y(x)}\approx \underbrace{-\frac{W}{k}}_{m}\cdot \underbrace{\frac1T}_{x}+\underbrace{\ln \left( \frac{p^2E}{3kT_p} \frac{N_p}{\tau_0}\right)}_{b}
\end{align}
Die Aktivierungsenergie $W$ kann somit über
\begin{align}
  W=-k\cdot m \label{eqn:Wcool}
\end{align}
berechnet werden.
Aus den lineare Regressionen folgt:
\begin{align*}
  m_1=-(4,51\pm0,23)\cdot10^{3}    &  &b_1=-(0,1\pm2,3)\cdot10^{2}\\
  m_2=-(2,99\pm0,21)\cdot10^{3}    &  &b_2=-(0,1\pm2,1)\cdot10^{2}
\end{align*}
Somit ergibt sich aus der Formel \eqref{eqn:Wcool} die Aktivierungsenergien:
\begin{align*}
  W_1=(0,39\pm0,02)\si{\electronvolt}\\
  W_2=(0,26\pm0,02)\si{\electronvolt}
\end{align*}

\subsection{Integral Methode}
Die Aktivierungsenergie kann ebenfall durch den Zusammmenhang \eqref{eqn:int}
berechnet werden.

\begin{align}
  \frac{W}{kT}+\ln{(const)}=\ln \frac{\int_T^{T^*}}{I(T')dT'}{I(T)} \label{eqn:int}
\end{align}

Wieder wird $1/T$ nun aber gegen
\begin{align}
  \ln \int_T^{T^*}I(T')dT'\frac{1}{I(T)} \label{eqn:intcool}
\end{align}
aufgetragen und eine
lineare Regression auf den Messwerten durchgeführt,
zu sehen in der Abblidung \ref{fig:intreg}.
Die linerare Regression vernachlässigt die Wertpaare
die zum Berechnen der Untergrundes genutzt wurden,
da diese zu stark schwanken.
Dabei ist $T^*$ ein feste Temperatur mit $I(T^*)\approx 0$.
Das Intergral in der Formel \ref{fig:intcool} von $T$ bis $T^*$
wird mit der Funktion trapz aus der Phython Bilbliotek scipy.integrate
berechnete.


\begin{figure}
  \centering
  \begin{subfigure}{0.48\textwidth}
    \centering
    \includegraphics[height=5.0cm]{build/plot5_1}
    \caption{1.Messung mit der Heizrate $1,2 \si{\kelvin\per\minute}$}
    \label{fig:intreg1}
  \end{subfigure}
  \begin{subfigure}{0.48\textwidth}
    \centering
    \includegraphics[height=5.0cm]{build/plot5_2}
    \caption{2.Messung mit der Heizrate $2 \si{\kelvin\per\minute}$.}
    \label{fig:intreg2}
  \end{subfigure}
\caption{Lineare Regression um $W$ aus dem gesamten Kurvenverlauf zu bestimmen.}
\label{fig:intreg}
\end{figure}

Aus den dieser lineare Regressionen folgt diesmal:
\begin{align*}
  m_1=(7,44\pm0,19)\cdot10^{3}    &  &b_1=-(0,3\pm1,9)\cdot10^{2}\\
  m_2=(7,18\pm0.27)\cdot10^{3}    &  &b_2=-(0.3\pm3,1)\cdot10^{2}
\end{align*}
Somit ergibt sich aus der Formel \eqref{eqn:int} die Aktivierungsenergien:
\begin{align*}
  W_1=(0,64\pm0,02)\si{\electronvolt}\\
  W_2=(0,62\pm0,02)\si{\electronvolt}
\end{align*}
