\section{Aufbau und Durchführung}
\label{sec:Durchführung}
Der Aufbau der Apparatur ist in Abbildung \ref{fig:aufbau}
zu finden. Die Probe, Kaliumbromid mit $0.005\%$ Mol Stronium, befindet sich im Kondensator,
dieser ist evakuiert.
Der Kondensator wird mit Gleichspannung versorgt und Strom mit einem Picoamperemeter gemessen.
Die Probe wird über den Kühlfinger geheizt, mittels Heizstromversorgung, oder mit flüssigem
Stickstoff gekühlt. Ein Thermoelement dient zur Temperaturmessung.
\begin{figure}
    \centering
    \includegraphics[width=0.7\textwidth]{aufbau.PNG}
    \caption{Aufbau der verwendeten Apparatur.\cite{skript}}
    \label{fig:aufbau}
\end{figure}\\
Zu Beginn wird der Kondensator mit $900\ \si{\volt}$ aufgeladen und die Probe auf $320\mathrm{K}$ erhitzt.
Beim Aufladen ist zu beachten, dass die Aufladezeit groß gegen die Relaxationszeit ist.
Anschließend wird die Probe, durch Eintauchen des Kühlfingers in Stickstoff,
auf $210\mathrm{K}$ gekühlt.
Das elektrische Feld wird nun abgestellt und der Kondensator zum Entladen kurzgeschlossen.
Zur Messung des Polarisationsstroms wird die Probe erneut auf $320\mathrm{K}$ erhitzt.
Beim Erhitzen werden in Abständen von einer Minute Strom-Temperatur-Paare mit dem Picoamperemeter aufgenommen.
Dieser Vorgang wird für zwei unterschiedliche Heizraten wiederholt.
