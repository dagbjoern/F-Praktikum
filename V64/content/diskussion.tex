\section{Diskussion}
\label{sec:Diskussion}
Die Messung der Brechungsindices von Glas
und Luft lieferte die folgenden Ergebnisse:
\begin{align*}
  n_\mathrm{Glas}=1,18\pm0,04\\
  n_\mathrm{Luft}=1,000264\pm 3\dot10^{-6}
\end{align*}
Werden diese mit den Literaturwerten
\begin{align*}
n_{\mathrm{lit}_\mathrm{Glas}}=1,45   \text{\cite{literatur}}\\
n_{\mathrm{lit}_\mathrm{Luft}}=1,000292    \text{\cite{literatur}}
\intertext{vergleicht, ergeben sich die entsprechenenden Abweichungen:}
a_\mathrm{Glas}\approx0,18\\
a_\mathrm{Luft}\approx2,8\cdot 10^{-5}.
\end{align*}
Die etwas höhere Abweichung bei dem Brechungsindex von Glas
im Gegensatzt zu Luft, lässt sich möglicherweise durch die
Nährung in der Formel
\eqref{eqn:glas} begrunden, da bei der Formel angenommen wird, dass nur ein Laserstrahl
eine Glasplatte passiert. Jedoch passieren in dem Experiment beide Laserstrahlen
zwei unterschiedlich ausgerichtete Glasplatten, wobei ein Winkel von einer Glasplatte gegen
$0 \si{\degree}$ geht. Folglich ist die eine Glasplatte nicht
vernachlässigbar und muss in der Formel berücksichtigt werden.
