\newpage
\section{Diskussion}
\label{sec:Diskussion}
Die Messung der Brechungsindices von Glas
und Luft liefert folgende Ergebnisse:
\begin{align*}
  n_\mathrm{Glas}&=1,18\pm0,04\\
  n_\mathrm{Luft}&=1,000264\pm 3\dot10^{-6}.
\end{align*}
Werden diese mit den Literaturwerten
\begin{align*}
n_{\mathrm{lit}_\mathrm{Glas}}&=1,5  \ \ \ \text{\cite{skript}}\\
n_{\mathrm{lit}_\mathrm{Luft}}&=1,000292     \ \ \ \text{\cite{literatur}}
\intertext{verglichen, ergeben sich die entsprechenden Abweichungen:}
a_\mathrm{Glas}&\approx0,06\\
a_\mathrm{Luft}&\approx2,8\cdot 10^{-5}.
\end{align*}
Die Messung des Brechungsindices
liefert recht genau Ergebnisse, was für die Präzision des Sagnac-Interferometer spricht.
Die etwas größere Abweichung bei dem Brechungsindex von Glas könnte daran liegen, dass
der Literaturwert von Glas nur ein Richtwert ist.  
