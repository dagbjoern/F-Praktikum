\section{Auswertung}
\label{sec:Auswertung}
\subsection{Kontrast}
\label{sec:Kontrast}
Zunächt muss der optimale Kontrast des Interfermometers bestimmt
werden.
Der Kontrast berechnete sich aus
\begin{align}
K=\frac{I_\mathrm{max}-I_\mathrm{min}}{I_\mathrm{max}+I_\mathrm{min}}\label{eqn:kontrast}
\end{align}
In der Tabelle \ref{tab:kontrast}
sind die entsprechenden Messwerte aufgelistet, der angegebene Winkel
entspricht dabei der Ausrichting des ersten Polarisationsfilters.
\begin{table}
\centering
\caption{Messwerte zur Berechnung des Kontrastes nach Formel \eqref{eqn:kontrast}.}
\label{tab:kontrast}
\begin{tabular}{c c c c c}
\toprule
   Winkel $\Phi/\si{\degree}$&   Winkel $\Phi/\si{\radian}$ &  Kontrast $K$ &   $I_\mathrm{max}/\si{\volt}$ &    $I_\mathrm{min}/\si{\volt}$ \\
\midrule
    -15 &   -0,262 &      0,403 &   0,634 &    0,27  \\
      0 &    0     &      0,1   &   0,622 &    0,509 \\
     15 &    0,262 &      0,288 &   0,871 &    0,481 \\
     30 &    0,524 &      0,626 &   1,064 &    0,245 \\
     45 &    0,785 &      0,826 &   1,026 &    0,098 \\
     60 &    1,047 &      0,8   &   0,882 &    0,098 \\
     75 &    1,309 &      0,505 &   0,474 &    0,156 \\
     90 &    1,571 &      0,123 &   0,292 &    0,228 \\
    105 &    1,833 &      0,366 &   0,263 &    0,122 \\
    120 &    2,094 &      0,703 &   0,293 &    0,051 \\
    135 &    2,356 &      0,831 &   0,412 &    0,038 \\
    150 &    2,618 &      0,742 &   0,566 &    0,084 \\
    165 &    2,88  &      0,436 &   0,674 &    0,265 \\
    180 &    3,142 &      0,047 &   0,601 &    0,547 \\
    195 &    3,403 &      0,369 &   0,906 &    0,418 \\
\bottomrule
\end{tabular}
\end{table}

Um nun den besten Winkel zu finden, bei dem der Kontrast maximal ist, wird
nun der Winkel gegen den Kontrast aufgetragen, zu sehen in der Abbildung \ref{fig:kontrast},
An den Messwerten wird nun die Funktion
\begin{align}
  K\left(\Phi\right) %= a \cdot | \sin\left(b\codt\Phi+c \right)| + d
\end{align}
gefittet.

\begin{figure}
    \centering
    \includegraphics[width=0.7\textwidth]{build/plot1.pdf}
    \caption{Verlauf des Kontrastest abhängig von dem Winkel des Polarisationsfilters.}
    \label{fig:kontrast}
\end{figure}

Es ergeben sich die folgenden Parameter:
\begin{align*}
  a&=0,83\pm0,03
  &b&=2,02\pm0,01\\
  c&=-0,12\pm0,03
  &d&=0\pm0,02
\end{align*}
Der beste Winkel ergibt sich aus einem belibigen
Hochpunkt der Funktion $K(\Phi)$ hierbei
\begin{align*}
 \Phi_\mathrm{max}\approx0,84\si{\radian}\approx48\si{\degree}.
\end{align*}
Bei den folgenden Messungen wird der erste Polarisationsfilter
nun immer auf den Winkel $\Phi=48\si{\degree}$ gestellt.


\subsection{Brechungsindex von Glas}
In der Tabelle \ref{tab:glas} ist die gemessene Anzahl der durchlaufenen Extrema $M$
und der aus Formel \eqref{eqn:glas} resultierenden Brechungsindex $n$ aufgetragen
für die unterschiedlichen Winkel $\theta$.

\begin{table}
  \begin{tabular}{c c | c c | c c | c c}
\caption{Zählrate der Extrema $M$ und die daraus nach Formel \eqref{eqn:glas} berechneten Brechungsindices $n$ für unterschiedliche Winkel $theta$}
\label{tab:glas}
    \toprule
\multicolumn{2}{c}{$\Theta=2\,\si{\degree}$}  &  \multicolumn{2}{c}{$\Theta=4\,\si{\degree}$} & \multicolumn{2}{c}{$\Theta=6\,\si{\degree}$} & \multicolumn{2}{c}{$\Theta=8\,\si{\degree}$}\\
      $M_2$ &  $n_2$ &  $M_4$ & $n_4$ & $M_6$ &  n\_6 & M_8 \\
  \midrule
       8 &  1,13 &  16 &  1,20 & 21 & 1,20 & 29 & 1,22\\
       8 &  1,13 &  17 &  1,22 & 23 & 1,22 & 30 & 1,23\\
       8 &  1,13 &  14 &  1,17 & 20 & 1,19 & 27 & 1,20\\
       7 &  1,11 &  14 &  1,17 & 22 & 1,21 & 27 & 1,20\\
       9 &  1,15 &  15 &  1,19 & 20 & 1,19 & 27 & 1,20\\
  \bottomrule
  \end{tabular}
\end{table}


\subsection{Brechungsindex von Luft}
Die Messwerte für den den Brechungsindex von Luft sind in der Tabelle \ref{tab:luft}
zu finden sowie die berechneten Brechungsindexes $n$ nach Formel \eqref{eqn:luft}.
Die Formel \eqref{eqn:luft} ergibt sich aus der Formel \eqref{eqn:gas}.
\begin{align}%    n=M*lam/L+1
    n=\frac{M\lambda_\mathrm{vac}{L+1}
\end{align}

\begin{table}
\centering
\caption{Messwerte zur Berechnung des Brechungsindexes $n$ und $n$ nach Formel \eqref{eqn:luft}.}
\label{tab:luft}
\begin{tabular}{c c }
\toprule}
  Anzahl der Fringes $M$ & Brechungsindex $n$ \\
\midrule
41 &  1,000260\\
42 &  1,000266\\
42 &  1,000266\\
\bottomrule
\end{tabular}
\end{table}
Durch Mittelung der berechneten Brechungsindexe aus der Tabelle \ref{tab:luft}
folgt:
\begin{align*}
  n_\mathrm{Luft}=1,000264\pm 3\dot10^{-6}
\end{align*}
