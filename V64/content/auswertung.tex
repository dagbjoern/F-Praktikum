\section{Auswertung}
\label{sec:Auswertung}
\subsection{Kontrast}
\label{sec:Kontrast}
Zunächt muss der optimale Kontrast des Interferometers bestimmt
werden.
Der Kontrast berechnete sich aus
\begin{align}
K=\frac{I_\mathrm{max}-I_\mathrm{min}}{I_\mathrm{max}+I_\mathrm{min}}\label{eqn:kontrast}.
\end{align}
In der Tabelle \ref{tab:kontrast}
sind die entsprechenden Messwerte aufgelistet, der angegebene Winkel
entspricht dabei der Ausrichtung des ersten Polarisationsfilters.
\begin{table}
\centering
\caption{Messwerte zur Berechnung des Kontrastes nach Formel \eqref{eqn:kontrast}.}
\label{tab:kontrast}
\begin{tabular}{c c c c c}
\toprule
   Winkel $\Phi/\si{\degree}$&   Winkel $\Phi/\si{\radian}$ &  Kontrast $K$ &   $I_\mathrm{max}/\si{\volt}$ &    $I_\mathrm{min}/\si{\volt}$ \\
\midrule
    -15 &   -0,262 &      0,403 &   0,634 &    0,270 \\
      0 &    0     &      0,100 &   0,622 &    0,509 \\
     15 &    0,262 &      0,288 &   0,871 &    0,481 \\
     30 &    0,524 &      0,626 &   1,064 &    0,245 \\
     45 &    0,785 &      0,826 &   1,026 &    0,098 \\
     60 &    1,047 &      0,800 &   0,882 &    0,098 \\
     75 &    1,309 &      0,505 &   0,474 &    0,156 \\
     90 &    1,571 &      0,123 &   0,292 &    0,228 \\
    105 &    1,833 &      0,366 &   0,263 &    0,122 \\
    120 &    2,094 &      0,703 &   0,293 &    0,051 \\
    135 &    2,356 &      0,831 &   0,412 &    0,038 \\
    150 &    2,618 &      0,742 &   0,566 &    0,084 \\
    165 &    2,88  &      0,436 &   0,674 &    0,265 \\
    180 &    3,142 &      0,047 &   0,601 &    0,547 \\
    195 &    3,403 &      0,369 &   0,906 &    0,418 \\
\bottomrule
\end{tabular}
\end{table}

Um den besten Winkel zu finden, bei dem der Kontrast maximal ist, wird
der Winkel gegen den Kontrast aufgetragen, zu sehen in der Abbildung \ref{fig:kontrast}.
An den Messwerten wird die Funktion
\begin{align}
  K\left(\Phi\right) = a \cdot \left| \sin\left(b\cdot\Phi+c \right)\right| + d
\end{align}
gefittet.

\begin{figure}
    \centering
    \includegraphics[width=0.7\textwidth]{build/plot1.pdf}
    \caption{Verlauf des Kontrastest abhängig von dem Winkel des Polarisationsfilters.}
    \label{fig:kontrast}
\end{figure}

Es ergeben sich die folgenden Parameter:
\begin{align*}
  a&=\num{0,83\pm0,03}
  &b&=\num{2,02\pm0,01}\\
  c&=\num{-0,12\pm0,03}
  &d&=\num{0\pm0,02}.
\end{align*}
Der beste Winkel ergibt sich aus einem beliebigen
Hochpunkt der Funktion $K(\Phi)$:
\begin{align*}
 \Phi_\mathrm{max}\approx  \SI{0,84(2)}{\radian} \approx \SI{48,1(9)}{\degree}.
\end{align*}
Bei den folgenden Messungen wird der erste Polarisationsfilter
nun immer auf den Winkel $\Phi=48\si{\degree}$ gestellt.


\subsection{Brechungsindex von Glas}
In der Tabelle \ref{tab:glas} ist die gemessene Anzahl der durchlaufenen Extrema $M$
%\eqref{eqn:glas}
für die unterschiedlichen Winkel $\theta$ und deren Mittelwerte aufgetragen.

\begin{table}
  \centering
  \caption{Gemessene Zählrate der Extrema $M$ für unterschiedliche Winkel $\theta$ und die Mittelwerte der Zählraten}
  \label{tab:glas}
  \begin{tabular}{c c c c}
    \toprule
$\Theta=2\,\si{\degree}$  &  $\Theta=4\,\si{\degree}$ & $\Theta=6\,\si{\degree}$ & $\Theta=8\,\si{\degree}$\\
      $M_2$ &   $M_4$ & $M_6$ &   $M_8$ \\
  \midrule
       8 &   16 &  21 & 29  \\
       8 &   17 &  23 & 30  \\
       8 &   14 &  20 & 27  \\
       7 &   14 &  22 & 27  \\
       9 &   15 &  20 &  27 \\
  \midrule
  Mittelwerte: & &  & \\
  8,0\pm0,6 & 15,2\pm1,2  & 21,2\pm1,2  & 28,0\pm1,3 \\
  \bottomrule
  \end{tabular}
\end{table}
Die Mittelwerte von $M$  aus Tabelle \ref{tab:glas} werden nun in Abhängigkeit von dem Winkel
$\theta$ aufgetragen, zusehen in der Abbildung \ref{fig:glas}


\begin{figure}
    \centering
    \includegraphics[width=0.7\textwidth]{build/plotglas.pdf}
    \caption{Anzahl der durchlaufenen Extrema $M$ in Abhängigkeit von dem Winkel $\theata$.}
    \label{fig:glas}
\end{figure}


Werden die berechneten Brechungsindices aus der Tabelle \ref{tab:glas} gemittelt folgt für den
bestimmten Brechungsindex für Glas
\begin{align*}
  n_\mathrm{Glas}=1,18\pm0,04.
\end{align*}


\subsection{Brechungsindex von Luft}
Die Messwerte für den Brechungsindex von Luft sind in der Tabelle \ref{tab:luft}
zu finden sowie die berechneten Brechungsindexes $n$ nach Formel \eqref{eqn:luft}.
Die Formel \eqref{eqn:luft} ergibt sich aus der Formel \eqref{eqn:gas}
\begin{align}%    n=M*lam/L+1
    n=\frac{M\lambda_\mathrm{vac}}{L+1} \label{eqn:luft}.
\end{align}

\begin{table}
\centering
\caption{Messwerte zur Berechnung des Brechungsindexes $n$ und $n$ nach Formel \eqref{eqn:luft}.}
\label{tab:luft}
\begin{tabular}{c c}
\toprule
  Anzahl der Fringe $M$ & Brechungsindex $n$ \\
\midrule
41 &  1,000260\\
42 &  1,000266\\
42 &  1,000266\\
\bottomrule
\end{tabular}
\end{table}
Durch Mittelung der berechneten Brechungsindices aus der Tabelle \ref{tab:luft}
folgt:
\begin{align*}
  n_\mathrm{Luft}=1,000264\pm 3\cdot10^{-6}.
\end{align*}
